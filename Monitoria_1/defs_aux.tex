% !TEX root = monitoria.tex
\documentclass[11pt]{article}
\usepackage[utf8]{inputenc}
\usepackage[letterpaper, includeheadfoot, top=0cm]{geometry}
\usepackage{amsmath,amssymb, amsthm}
\usepackage{mathtools}  
\usepackage{graphicx}
\usepackage{hyperref}
\usepackage{pdfpages}
\usepackage{fancyhdr}
\usepackage{hyperref}
\usepackage{titlesec}
\usepackage{mathrsfs}
\usepackage{float}
\usepackage{fullpage}
%TikZ
\usepackage{pgf,tikz}
\usepackage{stmaryrd}
\usetikzlibrary{arrows}
\usetikzlibrary[patterns]
\renewcommand{\labelenumi}{\normalsize\bfseries Q\arabic{enumi}.}
\renewcommand{\labelenumii}{\normalsize\bfseries (\alph{enumii})}
\renewcommand{\labelenumiii}{\normalsize\bfseries \roman{enumiii})}

% Heading
\newcommand{\enc}[3]{\Large \textbf{#1}\\ \normalsize #2\\ #3}

% Indentation
\setlength{\parindent}{0em}
\setlength{\parskip}{1ex}

\DeclareMathOperator{\Spec}{Spec}
\DeclareMathOperator{\proj}{Proj}
\DeclareMathOperator{\Hom}{Hom}
\DeclareMathOperator{\End}{End}
\DeclareMathOperator{\GL}{GL}
\DeclareMathOperator{\Aut}{Aut}
\newcommand{\MT}{\mathord{\text{MT}}}
\DeclareMathOperator{\im}{Im}
\DeclareMathOperator{\Gal}{Gal}
\DeclareMathOperator{\Lie}{Lie}
\DeclareMathOperator{\Res}{Res}
\DeclareMathOperator{\pr}{pr}
\newcommand{\Tl}{\mathord{\text{T}}_{\ell}}
\newcommand{\Vl}{\mathord{\text{V}}_{\ell}}

% Flechas y parentesis
%=========================================================
\newcommand{\ssi}{\Longleftrightarrow}
\newcommand{\imp}{\implies}
\newcommand{\pmi}{\Longleftarrow}
\newcommand{\contra}{\rightarrow \leftarrow}
\newcommand{\matriz}[1]{\begin{pmatrix} #1 \end{pmatrix}}
\newcommand{\hooklongrightarrow}{\lhook\joinrel\longrightarrow}
\newcommand{\hooklongleftarrow}{\longleftarrow\joinrel\rhook}
\newcommand{\embeds}{\hookrightarrow}
\newcommand{\Embeds}{\hooklongrightarrow}
\newcommand{\inv}{\text{inv}}
\newcommand{\To}{\longrightarrow}
%=========================================================

% Conjuntos usuales
%=========================================================
\newcommand{\N}{\mathbf{N}}
\newcommand{\Z}{\mathbf{Z}}
\newcommand{\Zl}{\Z_{\ell}}
\newcommand{\C}{\mathbf{C}}
\newcommand{\Cp}{\C_{p}}
\newcommand{\Cl}{\C_{\ell}}
\newcommand{\Q}{\mathbf{Q}}
\newcommand{\Qbar}{\overline{\Q}}
\newcommand{\Qp}{\Q_{p}}
\newcommand{\Qpbar}{\overline{\Qp}}
\newcommand{\Ql}{\Q_{\ell}}
\newcommand{\Qlbar}{\overline{\Ql}}
\newcommand{\R}{\mathbf{R}}
\renewcommand{\S}{\mathbb{S}}
\newcommand{\K}{\mathbf{K}}
\newcommand{\F}{\mathbf{F}}
\renewcommand{\P}{\mathbf{P}}
\newcommand{\T}{\mathbf{T}}
\newcommand{\Disc}{\mathbf{D}}
\newcommand{\D}{\mathbf{D}}
\newcommand{\vacio}{\varnothing}
\newcommand{\Rel}{\mathcal{R}}
\newcommand{\gen}[1]{\left \langle #1  \right \rangle }
\newcommand{\mbf}[1]{\mathbf{#1}}
\newcommand{\mfrak}[1]{\mathfrak{#1}}
%=========================================================

% Derivadas
%=========================================================
\newcommand{\der}[2]{\dfrac{\mathrm{d} #1}{\mathrm{d} #2}}
\newcommand{\ider}[2]{\dfrac{d #1}{d #2}}
\newcommand{\iider}[2]{\dfrac{d^2 #1}{d #2^2}}
\newcommand{\ipartial}[2]{\dfrac{\partial #1}{\partial #2}}
\newcommand{\iipartial}[2]{\dfrac{\partial^2 #1}{\partial #2^2}}
\newcommand{\ijpartial}[3]{\dfrac{\partial^2 #1}{\partial #2 \partial #3}}
\newcommand{\dd}{\mathrm{d}}
%=========================================================

% Letras
%=========================================================
\newcommand{\f}{\varphi}
\newcommand{\e}{\varepsilon}
%=========================================================

%Probabilidades
%=========================================================
\newcommand{\prob}{\mathrm{P}}
\newcommand{\Prob}{\mathbb{P}}
\newcommand{\Var}{\mathrm{Var}}
\newcommand{\Cov}{\mathrm{Cov}}
\newcommand{\Esp}{\mathbb{E}}
%=========================================================

%Miscelanea
%=========================================================
\newcommand{\Def}{\overset{\text{(def)}}{=}}
\newcommand{\nconj}[1]{\left \{  1, \ldots, #1   \right \}}
\newcommand{\Vector}[3]{ \Par{#1_{#3}, \ldots, #1 _{#2}} }
\newcommand{\ds}[1]{\displaystyle{#1}}
\newcommand{\normal}{\trianglelefteq}
\newcommand{\limite}{\lim\limits_{n\to \infty}}
\newcommand{\suma}[2]{\ds{\sum\limits_{#1}^{#2}}}
\newcommand{\Comp}[1]{\mathscr{O}\Par{#1}}
\newcommand{\rfrac}[2]{{}^{#1}\!/_{#2}}
\newcommand{\cod}[1]{\texttt{#1}}
\newcommand{\Id}{\text{Id}}
\newcommand{\units}[1]{#1^{\times}}
%=========================================================

% Algebra
%=========================================================
\newcommand{\module}{\text{-Mod}}
\newcommand{\frakprime}{\mathfrak{p}}
\newcommand{\Prime}{\mathfrak{P}}
\newcommand{\torPoint}[2]{\leftidx{_{#2}}{#1}{}}
%=========================================================

%Categories
%=========================================================
\newcommand{\catmod}[1]{\mathbf{#1}-\mathbf{Mod}}
\newcommand{\DiagGrp}[1][k]{\Par{\text{Diag. groups}/#1}}
\newcommand{\Tori}[1][k]{\Par{\text{Tori}/#1}}
\newcommand{\ZSigma}[1][k]{\Z\bra{\Sigma_{#1}}}

\DeclareFontFamily{OT1}{pzc}{}
\DeclareFontShape{OT1}{pzc}{m}{it}{<-> s * [1.2] pzcmi7t}{}
\DeclareMathAlphabet{\mathpzc}{OT1}{pzc}{m}{it}
\newcommand{\categ}{\mathpzc}
%=========================================================

%Geometria
%=========================================================
\newcommand{\Cinf}{\mathcal{C}^{\infty}}
\newcommand{\Cohom}[2]{H^{#1}\Par{#2}}
\newcommand{\formas}[2]{\Omega_{#2}^{#1}}
\newcommand{\dif}{\mathrm{d}}
\newcommand{\indice}[3]{#1_{#2_1}, \ldots, #1_{#2_{#3}}}
\newcommand{\seq}[2]{#1, \ldots, #2}
\newcommand{\sheaf}{\mathscr{S}}
\newcommand{\strshf}{\mathscr{O}}
\newcommand{\Hsing}[2]{H^{#1}_{\text{sing}}\Par{#2}}
\newcommand{\HdR}[2]{H^{#1}_{\text{dR}}\Par{#2}}
\newcommand{\Hetale}[2]{H^{#1}_{\text{ét}}\Par{#2}}
\newcommand{\Hladic}[2]{H^{#1}_{\ell}\Par{#2}}
%=========================================================

%=========================================================

%=============== ELLIPTIC CURVES =========================
\newcommand{\E}[1]{E\Par{#1}}
\newcommand{\Etor}[1]{E_{\text{tors}}\Par{#1}}
%=========================================================



%============THEOREM STYLE=============
% \theoremstyle{definition}

% %Definicion
\newtheorem{defi}{Definição}
%%\numberwithin{definition}{chapter}

% %Teorema 
% \newtheorem{thm}{Theorem}
% %%\numberwithin{theorem}{chapter}

% %Lema
\newtheorem{lemma}{Lema}
\newtheorem*{lemma*}{Lema}
% %%\numberwithin{lema}{chapter}

% %Corolario
\newtheorem{cor}{Corolário}
% %%\numberwithin{cor}{chapter}

% %Proposici�n
\newtheorem{prop}{Proposição}
% %%\numberwithin{prop}{chapter}

% %Propiedad
% \newtheorem*{prt}{Property}
% %%\numberwithin{prop}{chapter}

% %Problema
% \newtheorem*{prb}{Problem}
% %%\numberwithin{prop}{chapter}

% %Observacion
% \newtheorem*{rem}{Remark}
% %%\numberwithin{obs}{chapter}

% \newtheorem*{notation}{Notation}

% \newtheorem*{ex}{Exercise}
% \newtheorem{*example}{Example}
% % \numberwithin{ex}{chapter}


% \renewcommand{\qedsymbol}{\rule{0.7em}{0.7em}}
% \renewenvironment{proof}{{\bfseries \noindent Proof} }{ \qed \\}
%=====================================

% El comando \matriz funciona mejor que este.
%==================VECTOR COLUMNA=========================
% \newcount\colveccount 
% \newcommand*\colvec[1]{ 
%     \global\colveccount#1
%     \begin{pmatrix} 
%     \colvecnext 
% }
% \def\colvecnext#1{
%     #1
%     \global\advance\colveccount-1 
%     \ifnum\colveccount>0 
%         \\
%         \expandafter\colvecnext 
%     \else 
%         \end{pmatrix}
%     \fi
% }
%=========================================================


